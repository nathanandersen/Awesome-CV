%-------------------------------------------------------------------------------
%	SECTION TITLE
%-------------------------------------------------------------------------------
\cvsection{Work Experience}


%-------------------------------------------------------------------------------
%	CONTENT
%-------------------------------------------------------------------------------
\begin{cventries}
  \cventry
  {Qmulos}
  {Solutions Engineer}
  {Aug. 2020 - Present}
  {Boston, MA}
  {Training in Splunk and Q-Products.}
  {\begin{cvitems}
  \item Completed Splunk Core Consultant, Splunk Architect, Splunk Admin certifications.
\end{cvitems}
  }

  %---------------------------------------------------------
  \cventry
    {Akuna Capital}
    {Quantitative Developer (Jr.)}
    {Jul. 2018 - Jan. 2020} % Date(s)
    {Boston, MA} % Location
    {Specialized in Python dataset construction for this quantitative trading
    firm. Designed and built research datasets that improved access speeds up to
    100x over prior APIs. Also rebuilt live trading P\&L breakdown tool that
    expanded coverage and performance, from single-desk tool to high-visibility
    dashboard monitor for 100\% of traders. Enabled live access to more
    sophisticated analysis of trading patterns.}
    {\begin{cvitems} % Description(s) of tasks/responsibilities
      \item Collaborated with cross-function stakeholders (traders,
            researchers, developers) to design database schema, data
            generation applications, and access patterns.
      \item Implemented generic file-based Map-Reduce Python framework to
            parse daily trading parameter and market data into more easily
            accessible research formats.
%      \item Helped onboard Boston new-hires, acting as secondary resource
%            for trading theory and strategies. Held several small seminars and
%            weekly check-ins.
    \end{cvitems}}

% ---------------------------------------------------------
\cventry
{Andersen Mobile Labs} % Organization
{Founder \& IOS Engineer} % Job title
{Feb. 2016 - Jun. 2018} % Date(s)
{Williamstown, MA} % Location
{Conceived, built, launched, and iterated Eph Meals, iOS native (Swift) app for
Williams College Dining Services. Unlike previous access through
mobile-unfriendly website, this app featured simple access to daily menus. One
month post-release, 50\% of student body was using app; close to 100\% within
four months.}
{\begin{cvitems} % Description(s) of tasks/responsibilities
\item Met with dining services and college IT department to determine data
availability and optimal structure. Ultimately created native app around
undocumented API.
\item Marketed app through word-of-mouth referrals and college daily
newsletters.
%\item Engaged with other students weekly for feedback in iterative update
%process.
\end{cvitems}}

% ---------------------------------------------------------
\cventry
{Vistaprint US / Cimpress} % Organization
{Software Engineer Intern} % Job title
{May '15 - Sep. '15, Jun. '16 - Nov. '16} % Date(s)
{Waltham, MA} % Location
{Worked as software engineer at this custom-printing company. Built full-stack
design-auditing app that resulted in \$5M in cost savings by reducing
factory-printing mistakes.}
{\begin{cvitems} % Description(s) of tasks/responsibilities
\item Customer service agents used this app to identify products likely to
cause printing issues and enabled proactive customer service.
\item Met with various stakeholders – plant supervisors, customer service
agents, customer experience representatives – to gather requirements and design
app structure and interface.
\end{cvitems}}

% ---------------------------------------------------------
\cventry
{Williams College}
{Teacher's Assistant}
{2015-2017}
{Williamstown, MA}
{Worked one-on-one and in small groups to help students master introductory
computer science and more advanced probability courses. Focused on Python and
Java. Demonstrated cross-discipline importance of concepts, especially for
students pursuing degrees in other major disciplines.}
{}

%---------------------------------------------------------
\end{cventries}
